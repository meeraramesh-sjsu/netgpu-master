The importance of network security has been very crucial for the software industry. There is a need to speed up packet processing, hence several researchers have used GPUs(Graphics Processing Unit) for general purpose applications. The computationally intensive tasks from the CPU(Central Processing Unit) are offloaded to the GPU to exploit the power of the graphics card and hence improve the packet processing throughput. This thesis provides a comparison between three string matching algorithms for deep packet inspection on the GPU and the CPU. Comparisons have been made on the GPU with respect to the percentage of pipeline stalls, shared memory efficiency, warp efficiency and cache hits. The GPU version of the framework has been compared with the CPU version and OpenMP version of the framework. The general purpose GPU was used to perform header checking and pattern matching on the packets. Naive pattern algorithm, Rabin Karp pattern algorithm, Aho-Corasick algorithm and Wu Manber algorithm was explored. Two packet processing approaches are explored: thread level packet processing and block level packet processing. Evaluations were carried on a Tesla K80 GPU and we proved that block level packet processing outperforms thread level packet processing by a factor of 116. The packet processing was optimized to use shared memory and warp divergence has been eliminated. The results prove that GPUs can be utilized to speed up network traffic anomaly detection systems and in other systems that involve signature matching algorithms.
