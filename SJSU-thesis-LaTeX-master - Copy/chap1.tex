\chapter{Introduction}
\parindent 0.25in

Network intrusion detection systems (IDSs) such as firewalls capture malicious activities and drop malicious packets that intend to attack a network. These IDSs report the malicious activity to an administrator, and the administrator prevents the packets from moving across the network. There are two types of IDSs, host-based IDSs and network-based IDSs. Host-based IDSs monitor the operating system to see if it is being attacked, while network-based IDSs monitor network traffic. Network-based IDSs can detect malicious traffic in two ways by comparing against 1) malicious signatures and 2) reference traffic models. In this thesis, a new network-based IDS design is proposed that uses malicious signature analysis. To detect malicious traffic, individual packet contents were compared against well-known malicious packet signatures. 

To identify the existence of malicious signatures in the packet payload, various pattern-matching algorithms were used. Pattern matching is a computationally intensive task that accounts for up to 75\% of the execution time of IDSs \cite{bib6}. There is no dependency between instructions in signature matching. The text can be split into multiple chunks, and each chunk can be processed individually to search for patterns. This can be done in parallel with multiple threads. Thus, the massive parallelism of the GPU can be exploited in signature matching. A few studies have used various accelerators for DPI such as field programmable gate arrays (FPGAs) \cite{bib13}\cite{bib24}\cite{bib27} and graphics processing units (GPUs) \cite{bib12}\cite{bib25} for better performance. Hardware-based approaches using FPGAs may perform better than software approaches by optimizing the hardware pipeline dedicated to packet processing. However, they are not flexible enough to apply new signatures. Many researchers have used GPUs in various network security applications, including Gnort \cite{bib10} (a GPU-version of Snort \cite{bib11}) and Snap \cite{bib9} (a GPU version of clickOS \cite{bib7}).

In this study, I analyzed various approaches used to perform string matching on the GPU. I demonstrated the performance improvement by parallelizing the packet processing with multiple threads. I used various well-known signature-matching algorithms such as naive algorithm, the Rabin-Karp algorithm, the Aho-Corasick algorithm, and the Wu-Manber algorithm. The evaluation results were obtained by running single-pattern matching as well as multi-pattern matching on the GPU.

The proposed IDS consists of two parts, packet header processing and payload signature matching. Various rules were applied to check the integrity of the headers of various protocols such as internet protocol (IPv4) and transmission control protocol (TCP). Various optimizations were applied such as eliminating warp divergence, using pinned memory and using shared memory. To understand the performance difference of these algorithms, I characterized the architectural behavior of the algorithms by using two profilers, nvidia-smi and nvprof. 